\anonsection{Введение}

К числу достаточно новых геофизических технологий относится активный вибросейсмический мониторинг, 
он включает в себя методы по наблюдению и исследованию состояния земной коры по изменению 
различных характеристик вибро-сейсмических волн, порожденных некоторым вибрационным источником, 
и распространяющихся в некоторой среде. 

Само исследование процесса распространения упругих волн в неоднородных средах широко
используется при вибросейсмическом мониторинге различных геологических объектов.
Стоит отметить, что научные группы из Института
вычислительной математики и математической геофизики Сибирского
отделения Российской Академии наук (ИВМиМГ СО РАН) уже достаточно давно занимаются
задачами вибро-сейсмического мониторинга и имеют уникальный опыт в исследованиях.

В этом легко убедится на примерах работ Глинского Б.М, Ковалевского В.В и др.
по изучению грязевого вулкана «Гора Карабетова» \cite{glisnkiy-karabetov}, вулкана «Шуго» {} и т.д.

Саму задачу вибро-сейсмического мониторинга следует разделить на 2 больших класса: прямую и обратную.
При решении прямой задачи, при заданных параметрах среды, в которой распространяется волна, 
исследователи ставят своей целью рассчитать волновое поле. Обратная же задача, ставит своей целью найти параметры среды, 
в которых распространяется волна порожденная вибро-излучателем с заранее известными характеристиками. Часто реальные области исследования имеют довольно сложный рельеф,
который не позволяет расположить площадную систему наблюдения для корректной постановки обратной задачи. В связи с этим ее решение
обычно осуществляется с помощью решения набора прямых задач на основе проведения серии вычислительных
экспериментов с подбором значений параметров среды так, чтобы численное решение совпадало с результатами натурных экспериментов .

Для решения этих задач используются различные сеточные методы,  которые в зависимости от особенностей моделируемой среды позволяют получить решение с необходимой точностью.
В настоящее время в ИВМиМГ СО РАН накоплен опыт в создании алгоритмов и программ для решения таких задач \cite{konov, mih, ter, fat}.
Однако в связи с большими масштабами реальных задач и необходимостью, в первую очередь, решать обратную
задачу геофизики, через решение набора прямых задач, постоянно возникает необходимость в
разработке экономичных с точки зрения используемой памяти и времени вычислений
параллельных алгоритмов и программ, позволяющих с приемлемой точностью
моделировать распространение упругих волн в неоднородных средах на 
современных многоядерных вычислительных системах различной архитектуры.

С начала 70-х годов в ИВМиМГ СО РАН начал развиваться комбинированный подход, основанный на использовании аналитического метода разделения переменных в сочетании с конечно-разностным методом решения редуцированных одномерных задач \cite{alex}. Спектральные методы являются альтернативными по отношению к стандартным конечно-
разностным схемам для расчета сейсмических полей. Важным достоинством
спектральных методов является высокая скорость сходимости, если решение обладает
высокой степенью гладкости. Это позволяет получить хорошую точность взяв всего две-три
пространственные гармоники на минимальную длину волны, что значительно меньше, 
чем достаточное количество узлов на длину волны при применении конечно-разностного метода второго порядка точности.
Таким образом, можно получить экономию по памяти вычислительной системы, в сочетании с высокой точностью вычислений.

С учетом современного развития архитектуры ПЛИС и разработки эффективных реализаций алгоритма быстрого преобразоваия Фурье (БПФ), дополнительную актуальность получает разработка с спектральных методов на основе разложения Фурье.

В данной работе рассматривается 2D спектрально-разностный метод, основанный на
объединении конечно-разностного метода по одной координате и конечного
преобразования Фурье по другой. Возникающие при этом суммы
типа свертки вычисляются с помощью быстрого преобразования Фурье (БПФ).

При использовании такого подхода для сред с разрывными параметрами возникает явление Гиббса,
которое можно устранить, предварительно фильтруя и сглаживая разрывные функции,
чтобы получить решение сравнимое с конечно-разностным.

Таким образом, целью работы является разработка  и исследование спектрально-разностного
параллельного алгоритма и программного пакета на его основе для моделирования распространения
упругих волн в 2D неоднородных средах.

Для достижения этой цели были поставлены и выполнены следующие задачи:
\begin{itemize}
    \item разработать параллельный алгоритм и программное обеспечение, реализующее спектрально-
разностный метод и эффективно использующее современную вычислительную
архитектуру многоядерного центрального процессора (CPU);
    \item на примере выбранного алгоритма исследовать особенности оптимизации и адаптации программ под архитектуру многоядерных CPU;
    \item исследовать время работы и масштабируемость разработанного ПО в
сравнении с уже имеющимися программами, реализующими конечно-разностную схему
Верье.
\end{itemize}

Работа предполагает оригинальное развитие известного спектрально-разностного подхода
на основе конечного преобразования Фурье к моделированию упругих волн в
неоднородных средах, который может стать альтернативой стандартным конечно-разностным схемам.

Отдельно отметим практическую значимость разработки таких подходов в связи с развитием
программируемых логических схем (ПЛИС), выполняющих БПФ за минимальное время.

\clearpage
