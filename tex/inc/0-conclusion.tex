\anonsection{Заключение}

В результате выполнения квалификационной работы бакалавра был реализован и изучен спектрально-разностный алгоритм,
для вычисления сейсмических полей в задачах геофизики.

Разработано последовательное и параллельное прогрманное обеспечение(ПО), эффективно реализующие спектрально-
разностный метод на современных многоядерных вычислительных системах.

Исследованы особенности оптимизации и адаптации под архитектуру многоядерных CPU при реализации спектрально-разностного метода. Проведено тестирование времени работы и масштабируемости разработанного ПО, как на персональных компьютерах,
так и на специализированных вычислительных системах ЦКП ССКЦ ИВМиМГ СО РАН, показавшие эффективность распараллеливания.

Реализовано сравнение исследуемого спектрального-разностного подхода с сопоставимым ему конечно-разностным.

На защиту выносятся следующие пункты:

\begin{itemize}
    \item Реализация последовательного алгоритма для решения 2D задачи о моделировании
    распространения плоских упругих волн на языке С++.
    \item Реализация параллельного спектрально-разностного алгоритма для систем с общей памятью средствами OpenMP. В том числе
    результаты исследований ускорения исполнения параллельного кода с использованием SIMD расширений современных процессоров.
    \item Реализация параллельного алгоритма для системы фрагментированного программирования LuNA
    \item Тестирование и анализ разработанного ПО и полученных с его помощью результатов.
\end{itemize}

В дальнейшие планы входит более подробное исследование поведения рассматриваемого алгоритма в гетерогенных средах,
а также подбор низкочастотных фильтров для необходимой предобработки параметров среды с целью улучшения качества получаемого в результате расчетов решения.
Также будет продолжена работа над эффективной реализацией параллельной программы
средствами системы фрагментированного программирования LuNA.

\clearpage
