\anonsection{Заключение}

В результате выполнения квалификационной работы бакалавра был реализован и изучен спектрально-разностный алгоритм,
для вычисления сейсмических полей в задачах геофизики.

Разработан параллельный алгоритм и последовательное и параллельное программное обеспечение(ПО), эффективно реализующие спектрально-
разностный метод на современных многоядерных вычислительных системах.

Исследованы особенности оптимизации и адаптации под архитектуру многоядерных CPU при реализации спектрально-разностного метода. Проведено тестирование времени работы и масштабируемости разработанного ПО, как на персональных компьютерах,
так и на специализированных вычислительных системах ЦКП ССКЦ ИВМиМГ СО РАН, показавшие эффективность распараллеливания.

Реализовано сравнение времени расчетов на основе исследуемого спектрального-разностного подхода с расчетами с использованием конечно-разностной схемы Верье. 



В дальнейшие планы входит более подробное исследование поведения рассматриваемого алгоритма в гетерогенных средах,
а также подбор низкочастотных фильтров для необходимой предобработки параметров среды с целью улучшения качества получаемого в результате расчетов решения.
Также будет продолжена работа над эффективной реализацией параллельной программы
средствами системы фрагментированного программирования LuNA.

\clearpage
