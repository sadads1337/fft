\begin{singlespacing} % На титульнике одинарный интервал
\thispagestyle{empty} % Не нумеруем титульный лист
\begin{center}
\large{\textbf{Аннотация}}\\
\end{center} 
\end{singlespacing}
\begin{singlespacing}
К числу достаточно новых геофизических технологий относится активный вибросейсмический мониторинг, 
он включает в себя методы по наблюдению и исследованию состояния земной коры по изменению 
различных характеристик вибро-сейсмических волн, порожденных некоторым вибрационным источником, 
и распростроняющихся в некоторой среде. Само исследование процесса распространения упругих волн в неоднородных средах широко
используется при вибросейсмическом мониторинге различных геологических объектов.

Для решения этих задач используются различные сеточные методы,  которые в зависимости от особенностей моделируемой среды позволяют получить решение с необходимой точностью. Однако в связи с большими масштабами реальных задач и необходимостью, в первую очередь, решать обратную
задачу геофизики, через решение набора прямых задач, постоянно возникает необходимость в
разработке экономичных с точки зрения используемой памяти и времени вычислений
параллельных алгоритмов и программ, позволяющих с приемлемой точностью
моделировать распространение упругих волн в неоднородных средах на 
современных многоядерных вычислительных системах различной архитектуры.

Спектральные методы являются альтернативными по отношению к стандартным конечно-
разностным схемам для расчета сейсмических полей. Работа предполагает оригинальное развитие известного спектрально-разностного подхода
на основе конечного преобразования Фурье к моделированию упругих волн в
2D неоднородных средах, который может стать альтернативой стандартным конечно-разностным схемам.

Таким образом, целью работы является разработка  и исследование спектрально-разностного
параллельного алгоритма и программного пакета на его основе для моделирования распространения
упругих волн в 2D неоднородных средах.

Для достижения этой цели были поставлены и выполнены следующие задачи:
\begin{itemize}
    \item Разработан параллельный алгоритм и программное обеспечение, реализующее спектрально-
разностный метод и эффективно использующее современную вычислительную
архитектуру многоядерного центрального процессора;
    \item На примере выбранного алгоритма исследованы особенности оптимизации и адаптации программ под архитектуру многоядерных ЦП;
    \item Проанализированы время работы и масштабируемость разработанного ПО в
сравнении с уже имеющимися программами, реализующими конечно-разностную схему Верье;
\end{itemize}

\end{singlespacing}

\clearpage