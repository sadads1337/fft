\begingroup 
\renewcommand{\section}[2]{\anonsection{Библиографический список}}
\begin{thebibliography}{00}

\bibitem{alex}
	Alekseev A.S., Mikhailenko B.G.
	THE SOLUTION OF DYNAMIC PROBLEMS OF ELASTIC WAVE PROPAGATION IN INHOMOGENEOUS MEDIA BY A COMBINATION OF PARTIAL SEPARATION OF VARIABLES AND FINITE DIFFERENCE METHODS // International Journal of Geophysics. 1980. Vol. 48. P. 161­172

\bibitem{intel}
	Intel Россия
    Каталог программных продуктов семейства Intel [Электронный ресурс] //
    URL: https://intel.com

\bibitem{karavaev}
    Караваев Д. А.
    Чиссленное моделирование 3D волновых полей в задачах сейсмического зондирования вулканических структур//
    Диссертация на соискание ученой степени кандидата физико-математических наук. Новосибирск, 2011. С. 19-21.

\bibitem{glisnkiy-karabetov}
    Глинский Б. М., Караваев Д. А., Ковалевский В. В., Мартынов В. Н.
    Численное моделирование и экспериментальные исследования грязевого вулкана «Гора Карабетова» вибросейсмическими методами //
    Вычислительные методы и программирование. Москва, 2010.

\bibitem{mart}
    Мартынов В. Н.
    Волновые поля от сосредоточенных источников в трансверсально-изотропных средах //
    Физика земли. № 11, 1986.

\bibitem{novac}
    Новацкий В.
    Теория упругости //
    Москва, Мир,1975, 872с.

\bibitem{filon}
	Филоненко-Бородич М.М.
	Теория упругости //
	Государственное издательство физико-математической литературы, Москва, 1959, 365 с.

\bibitem{shem}
	Шемякин Е.И.
	Динамические задачи теории упругости и пластичности //
	Курс лекций для студентов. - НГУ, Новосибирск, 1968. -336 с.

\bibitem{mach}
	С.В. Мачульскис
	Разработка и реализация алгоритмов статического анализа фрагментированных программ //
	Квалификационная работа, Новосибисрк, 2015, С. 16-17

\end{thebibliography}
\endgroup

\clearpage