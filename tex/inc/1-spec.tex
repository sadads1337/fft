\section{Спектрально-разностное моделирование распространения сейсмических волн в упругих средах}

В данной главе описана общая математическая модель для решения численной задачи о распространении
упругих волн в 2D неоднородных средах, которые возникают в результате воздействия источника,
расположенного непосредственно в самой исследуемой среде (вблизи ее поверхности).

\subsection{Постановка задачи}
Исследование распространения упругих сейсмических волн в неоднородных средах будет проводится на 
основе 2D модели теории упругости с соответсвующими начальными и граничными условиями \cite{novac, filon, shem}

В качестве среды рассматривается изотропная 2D неоднородная среда, представляющая собой прямоугольник, размера $a \times b$, $a, b>0$,
одна из границ (плоскость $z = 0$) которого является свободной поверхностью.

Прямоугольная декартова система координат введена таким образом, чтобы ось $Oz$ была направлена вертикально вниз,
а ось $Ox$ лежала на свободной поверхности.

\addimghere{env}{0.30}{Система координат}{env}

Задача записывается в терминах вектора скоростей смещений $\vec{u}={(u, w)}^T$
и тензора напряжений $\vec{\sigma}=(\sigma_{xx}, \sigma_{xz}, \sigma_{zz})^T$ в следующем виде.

\begin{equation}
\label{eq:1}
\begin{dcases}
	\rho\frac{\partial{\vec{u}}}{\partial{t}}
	=\begin{bmatrix}
    \frac{\partial{}}{\partial{x}} & \frac{\partial{}}{\partial{z}} & 0 \\
    0 & \frac{\partial{}}{\partial{x}} & \frac{\partial{}}{\partial{z}}
	\end{bmatrix}
	\vec{\sigma} + f(x{,} z{,} t), \\
	\frac{\partial{\vec{\sigma}}}{\partial{t}}
	=\begin{bmatrix}
    (\lambda+2\mu)\frac{\partial{}}{\partial{x}} & \lambda\frac{\partial{}}{\partial{z}} \\
    \mu\frac{\partial{}}{\partial{z}} & \mu\frac{\partial{}}{\partial{x}} \\
    \lambda\frac{\partial{}}{\partial{x}} & (\lambda+2\mu)\frac{\partial{}}{\partial{z}}
	\end{bmatrix}
	\vec{u},
\end{dcases}
\end{equation}
где $t$ - время, $u$, $w$ - компоненты скоростей смещений по $Ox$ и $Oz$ соотвественно.
Параметры $\lambda(x,z), \mu(x,z)$ - коэффициенты Ламе, удовлетворяющие соотношениям: $\lambda + 2\mu=\rho{v}_p^2$, $\mu=\rho{v}_s^2$, 
где $v_p$ - скорость распространения продольных волн, $v_s$ - скорость распространения поперечных волн,
а $\rho(x,z)>0, \forall x,z$ - плотность среды.

Граничные условия на свободной поверхности (плоскость $z = 0$) задаются в виде:

\begin{equation}
	\label{eq:2}
	\sigma_{zz}|_{z=0}=\sigma_{xz}|_{z=0}=0
\end{equation}

В качестве начальных условий, в момент времени $t=0$ положим:

\begin{equation}
\begin{dcases}
	\label{eq:3}
	u |_{t=0}= w|_{t=0} = 0, \\
	\sigma_{xx}|_{t=0}=\sigma_{xz}|_{z=0}=\sigma_{zz}|_{z=0}=0
\end{dcases}
\end{equation}

Также предполагается, что функция источника представима в виде $f(x,z,t)=f_xi+f_zk$,
где $i, k$ - единичные направляющие вектора соотвествующих координатных осей.

В случае точечного источника типа "центр давления", расположенного в среде, функция $f(x,z,t)$ примет вид $f(x,z,t)=\delta(x-x_0)\delta(z-z_0)f(t)$,
где $\delta$ -  дельта функция Дирака, а $x_0, z_0$ - координаты источника.
\addimghere{env-source}{0.30}{Положение источника}{env-source}
Представленная система уравнений (\ref{eq:1}), в совокупности с начальными и граничными условиями (\ref{eq:2}-\ref{eq:3}),
описывает распространения упругих волн в неоднородной среде с точечным истоником.
\subsection{Метод решения: спектральное представление}
В отличии от работы \cite{karavaev} метод расчета сейсмических полей для задачи (\ref{eq:1}-\ref{eq:3}) основан на альтернативном,
не столь популярном, как его конечно-разностные аналоги, спектрально-разностном подходе. Этот метод обладает высокой скоростью сходимости,
при достаточно гладких решениях исходной задачи, что позволяет получить решение сравнимое по точности со стандартным
конечно-разностным методом.
Также в спектрально-разностном методе происходит автоматический учет характера неоднородности среды по горизонтальным переменным при переходе от одной точки разностной схемы к другой, и в зависимости от этого меняется трудоемкость вычислений. Такой подход позволяет эффективно вычислять теоретические сейсмограммы и численные снимки для сложнопостроенных моделей реальных сред.
Рассмотрим следующие прямые преобразования фурье для функций $u$, $w$, $\sigma_{xx}$, $\sigma_{xz}$, $\sigma_{zz}$.
\begin{equation}
\begin{sqcases}
	\label{eq:4}
	\overline{u}_{\overline k} = \int_0^b{u \cdot sin(\overline k x)} dx = \int_0^b{u(x,z,t) \cdot sin(\overline k x)} dx = \overline{u}_{\overline k}(x,z,t), \\
	\overline{w}_{\overline k} = \int_0^b{w \cdot cos(\overline kx)} dx = \int_0^b{w(x,z,t) \cdot cos(\overline kx)} dx = \overline{w}_{\overline k}(x,z,t), \\
	\overline{p}_{\overline k} = \overline{\sigma}_{xx}^{\overline k} = \int_0^b{{\sigma}_{xx} \cdot cos(\overline kx)} dx 
		= \int_0^b{{\sigma}_{xx}(x,z,t) \cdot cos(\overline kx)} dx = \overline{{\sigma}}_{xx}^{\overline k}(x,z,t), \\
	\overline{q}_{\overline k} = \overline{\sigma}_{xz}^{\overline k} = \int_0^b{{\sigma}_{xz} \cdot sin(\overline kx)} dx 
		= \int_0^b{{\sigma}_{xz}(x,z,t) \cdot sin(\overline kx)} dx = \overline{{\sigma}}_{xz}^{\overline k}(x,z,t), \\
	\overline{s}_{\overline k} = \overline{\sigma}_{zz}^{\overline k} = \int_0^b{{\sigma}_{zz} \cdot cos(\overline kx)} dx 
		= \int_0^b{{\sigma}_{zz}(x,z,t) \cdot cos(\overline kx)} dx = \overline{{\sigma}}_{zz}^{\overline k}(x,z,t)
\end{sqcases}
\end{equation}
Где полагаем, что $\overline k=\frac{k\pi}{b}$.

2D cпектрально-разностный метод для решения задачи (\ref{eq:1}-\ref{eq:3}), основанный на объединении
конечно-разностного метода по $z$ координате и конечного преобразования Фурье по $x$ координате
наиболее подробно описан в работе \cite{mart}. Ниже приведем только основные выкладки.

Умножим каждое уравнение системы (\ref{eq:1}) на соответсвующую базисную $k$-функцию и проинтегрируем
по $x$, предполагая, что
$$
\frac{1}{\rho(x,z,t)} = \frac{\overline{\rho}_0}{2} + \sum_{\overline l=1}^\infty{\overline{\rho}_{\overline l} \cdot cos(\overline lx)}
	= \frac{\overline{\rho}_0(0,z)}{2} + \sum_{\overline l=1}^\infty{\overline{\rho}_{\overline l}(\overline l,z) \cdot cos(\overline lx)}
$$
$$
(\lambda + 2 \mu)(x,z) = \frac{(\lambda + 2 \mu)_0}{2} + \sum_{\overline l=1}^\infty{(\overline{\lambda} + 2 \overline{\mu})_{\overline l} \cdot cos(\overline lx)}
$$
$$
\mu(x,z) = \frac{\overline{\mu}_0}{2} + \sum_{\overline l=1}^\infty{\overline{\mu}_l \cdot cos(\overline lx)}
$$
$$
\lambda(x,z) = \frac{\overline{\lambda}_0}{2} + \sum_{\overline l=1}^\infty{\overline{\lambda}_0 \cdot cos(\overline lx)}
$$

Использая формулу интегрирования по частям, формулу произведения тригонометрических функций, приводя подобные слагаемые, и замену (\ref{eq:4})
Получим следующее соотношение.
\begin{longaligned}
\label{eq:5}
{}
&\int_0^b\frac{1}{\rho}\frac{\partial\sigma_{xx}}{\partial x}sin(\overline kx)dx
=\int_0^b\frac{\partial\sigma_{xx}}{\partial x} \left( \frac{1}{\rho}sin(\overline kx)dx \right) = \\
&=\underbrace{{\left[\sigma_{xx} \frac{1}{\rho} sin(\overline kx) \right]}_0^b}_{=0} 
	- \int_0^b\sigma_{xx}\frac{\partial}{\partial x}\left[ \frac{1}{\rho}sin(\overline kx)\right]dx =\\
&= - \int_0^b\sigma_{xx}\frac{\partial}{\partial x}\left[ 
	\left(\ \frac{\overline{\rho}_0}{2} + \sum_{\overline l=1}^\infty{\overline{\rho}_{\overline l} \cdot cos(\overline lx)} \right)sin(\overline kx)\right]dx =\\
&= - \int_0^b\sigma_{xx}\frac{\partial}{\partial x}\left[ 
	\frac{\overline{\rho}_0}{2}sin(\overline kx) + \sum_{\overline l=1}^\infty{\overline{\rho}_{\overline l} \cdot cos(\overline lx)} sin(\overline kx)\right]dx =\\
&= - \int_0^b\sigma_{xx}\frac{\partial}{\partial x}\left[ 
	\frac{\overline{\rho}_0}{2}sin(\overline kx) + \frac{1}{2}\sum_{\overline l=1}^\infty{\overline{\rho}_{\overline l} \cdot sin\left((\overline k+\overline l)x\right)} \right. +\\
		&\left.+ \frac{1}{2}\sum_{\overline l=1}^\infty{\overline{\rho}_{\overline l} \cdot sin\left((\overline k-\overline l)x\right)} \right]dx = \longalignedtag\\
&= - \int_0^b\sigma_{xx}\left[ 
	k\frac{\overline{\rho}_0}{2}cos(\overline kx) + \frac{1}{2}\left(\sum_{\overline l=1}^\infty{(\overline k+\overline l)\overline{\rho}_{\overline l} \cdot cos\left((\overline k+\overline l)x\right)} \right.\right. +\\
		&\left.\left.+ \sum_{\overline l=1}^\infty{(\overline k-\overline l)\overline{\rho}_l \cdot cos\left((\overline k-\overline l)x\right)}\right)\right]dx =\\
&= -\overline k \overline{\sigma}_{xx}^{\overline k} \frac{\overline{\rho}_0}{2} - \frac{1}{2}\left(
	\sum_{\overline l=1}^\infty{(\overline k+\overline l)\overline{\rho}_{\overline l} \cdot \overline{\sigma}_{xx}^{(\overline k+\overline l)}}
		+ \sum_{l=1}^\infty{(k-l)\overline{\rho}_l \cdot \overline{\sigma}_{xx}^{(k-l)}}\right) =\\
&= - \frac{1}{2} \cdot \left(
	\sum_{\overline l=0}^\infty{(\overline k+\overline l)\overline{\rho}_{\overline l} \cdot \overline{\sigma}_{xx}^{(\overline k+\overline l)}}
		+ \sum_{\overline l=0}^\infty{(\overline k-\overline l)\overline{\rho}_{\overline l} \cdot \overline{\sigma}_{xx}^{(\overline k-\overline l)}}\right) =\\
&= - \frac{1}{2} \cdot \left(
	\sum_{\overline l=0}^\infty{(\overline k+\overline l)\overline{\rho}_{\overline l} \cdot \overline{p}^{(\overline k+\overline l)}}
		+ \sum_{\overline l=0}^\infty{(\overline k-\overline l)\overline{\rho}_{\overline l} \cdot \overline{p}^{(\overline k-\overline l)}}\right)
\end{longaligned}

Аналогичным образом получаем оставшиеся соотношения.
\begin{longaligned}
\label{eq:6}
{}
&\int_0^b{\frac{1}{\rho}\frac{\partial\sigma_{xz}}{\partial z}sin(\overline kx)dx} =\\
&= \frac{1}{2} \cdot \left(
	\sum_{\overline l=0}^\infty{(\overline k+\overline l)\overline{\rho}_{\overline l} \cdot \frac{\partial \overline{\sigma}_{xz}^{(\overline k+\overline l)}}{\partial z}} \right.+
		\left.+ \sum_{\overline l=0}^\infty{(\overline k-\overline l)\overline{\rho}_{\overline l} \cdot \frac{\partial\overline{\sigma}_{xz}^{(\overline k-\overline l)}}{\partial z}}\right) =\\
&= \frac{1}{2} \cdot \left(
	\sum_{\overline l=0}^\infty{(\overline k+\overline l)\overline{\rho}_{\overline l} \cdot \frac{\partial \overline{q}^{(\overline k+\overline l)}}{\partial z}}
		+ \sum_{\overline l=0}^\infty{(\overline k-\overline l)\overline{\rho}_{\overline l} \cdot \frac{\partial\overline{q}^{(\overline k-\overline l)}}{\partial z}}\right); \\
&\int_0^b{\frac{1}{\rho}\frac{\partial\sigma_{xz}}{\partial x}cos(\overline kx)dx} =\\
&= \frac{1}{2} \cdot \left(
	\sum_{\overline l=0}^\infty{(\overline k+\overline l)\overline{\rho}_{\overline l} \cdot \overline{\sigma}_{xz}^{(\overline k+\overline l)}}
		+ \sum_{\overline l=0}^\infty{(\overline k-\overline l)\overline{\rho}_{\overline l} \cdot \overline{\sigma}_{xz}^{(\overline k-\overline l)}}\right) =\\
&= \frac{1}{2} \cdot \left(
	\sum_{\overline l=0}^\infty{(\overline k+\overline l)\overline{\rho}_{\overline l} \cdot \overline{q}^{(\overline k+\overline l)}}
		+ \sum_{\overline l=0}^\infty{(\overline k-\overline l)\overline{\rho}_{\overline l} \cdot \overline{q}^{(\overline k-\overline l)}}\right); \\
&\int_0^b{\frac{1}{\rho}\frac{\partial\sigma_{zz}}{\partial z}cos(kx)dx} = \longalignedtag\\
&= \frac{1}{2} \cdot \left(
	\sum_{\overline l=0}^\infty{(\overline k+\overline l)\overline{\rho}_{\overline l} \cdot \frac{\partial \overline{\sigma}_{zz}^{(\overline k+\overline l)}}{\partial z}}
		+ \sum_{\overline l=0}^\infty{(\overline k-\overline l)\overline{\rho}_{\overline l} \cdot \frac{\partial\overline{\sigma}_{zz}^{(\overline k-\overline l)}}{\partial z}}\right) =\\
&= \frac{1}{2} \cdot \left(
	\sum_{\overline l=0}^\infty{(\overline k+\overline l)\overline{\rho}_{\overline l} \cdot \frac{\partial \overline{s}^{(\overline k+\overline l)}}{\partial z}}
		+ \sum_{\overline l=0}^\infty{(\overline k-\overline l)\overline{\rho}_{\overline l} \cdot \frac{\partial\overline{s}^{(\overline k-\overline l)}}{\partial z}}\right); \\
&\int_0^b{(\lambda + 2 \mu) \frac{\partial u}{\partial x}cos(\overline kx)dx} =\\
&= \frac{1}{2} \cdot \left(
	\sum_{\overline l=0}^\infty{(\overline k+\overline l)(\overline{\lambda}_{\overline l} + 2\overline{\mu}_{\overline l}) \cdot \overline{u}^{(\overline k+\overline l)}}
		+ \sum_{\overline l=0}^\infty{(\overline k-\overline l)(\overline{\lambda}_{\overline l} + 2\overline{\mu}_{\overline l}) \cdot \overline{u}^{(\overline k-\overline l)}}\right); \\
&\int_0^b{\lambda \frac{\partial w}{\partial z}cos(\overline kx)dx}
= \frac{1}{2} \cdot \left(
	\sum_{\overline l=0}^\infty{(\overline k+\overline l)\overline{\lambda}_{\overline l} \cdot \frac{\partial\overline{w}^{(\overline k+\overline l)}}{\partial z}}
		+ \sum_{\overline l=0}^\infty{(\overline k-\overline l)\overline{\lambda}_{\overline l} \cdot \frac{\partial\overline{w}^{(\overline k-\overline l)}}{\partial z}}\right); \\
&\int_0^b{\lambda \frac{\partial u}{\partial z}sin(\overline kx)dx}
= \frac{1}{2} \cdot \left(
	\sum_{\overline l=0}^\infty{(\overline k+\overline l)\overline{\mu}_{\overline l} \cdot \frac{\partial\overline{u}^{(\overline k+\overline l)}}{\partial z}}
		+ \sum_{\overline l=0}^\infty{(\overline k-\overline l)\overline{\mu}_{\overline l} \cdot \frac{\partial\overline{u}^{(\overline k-\overline l)}}{\partial z}}\right); \\
&\int_0^b{\mu \frac{\partial w}{\partial x}sin(\overline kx)dx}
= -\frac{1}{2} \cdot \left(
	\sum_{\overline l=0}^\infty{(\overline k+\overline l)\overline{\mu}_{\overline l} \cdot \overline{w}^{(\overline k+\overline l)}}
		+ \sum_{\overline l=0}^\infty{(\overline k-\overline l)\overline{\mu}_{\overline l} \cdot \overline{w}^{(\overline k-\overline l)}}\right); \\
&\int_0^b{(\lambda + 2 \mu) \frac{\partial w}{\partial z}cos(\overline kx)dx} =\\
&= \frac{1}{2} \cdot \left(
	\sum_{\overline l=0}^\infty{(\overline k+\overline l)(\overline{\lambda}_{\overline l} + 2\overline{\mu}_{\overline l}) \cdot \frac{\overline{w}^{(\overline k+\overline l)}}{\partial z}}
		+ \sum_{\overline l=0}^\infty{(\overline k-\overline l)(\overline{\lambda}_{\overline l} + 2\overline{\mu}_{\overline l}) \cdot \frac{\overline{w}^{(\overline k-\overline l)}}{\partial z}}\right); \\
&\int_0^b{\lambda \frac{\partial u}{\partial x}cos(\overline kx)dx}
= -\frac{1}{2} \cdot \left(
	\sum_{\overline l=0}^\infty{(\overline k+\overline l)\overline{\lambda}_{\overline l} \cdot \overline{u}^{(\overline k+\overline l)}}
		+ \sum_{\overline l=0}^\infty{(\overline k-\overline l)\overline{\lambda}_{\overline l} \cdot \overline{u}^{(\overline k-\overline l)}}\right);
\end{longaligned}

Таким образом, используя (\ref{eq:5}) и (\ref{eq:6}), и обозначая 
$$conv(x,y)_{\overline k} = \sum_{\overline l=0}^\infty(\overline k-\overline l)x^{(\overline k-\overline l)}y^{\overline l}$$
$$corr(x,y)_{\overline k} = \sum_{\overline l=0}^\infty(\overline k+\overline l)x^{(\overline k+\overline l)}y^{\overline l}$$
$$sum(x,y)_{\overline k} = conv(x,y)_{\overline k} + corr(x,y)_{\overline k}$$
можно привести исходную систему (\ref{eq:1}-\ref{eq:3}) к виду.
\begin{longaligned}
\label{eq:7}
{}
&\frac{\partial\overline{u}_{\overline k}}{\partial t} = \frac{1}{2} \cdot \sum_{\overline l=0}^\infty\left(
	-(\overline k+\overline l)\overline{\rho}_{\overline l}\overline{p}^{(\overline k+\overline l)}
		-(\overline k-\overline l)\overline{\rho}_{\overline l}\overline{p}^{(\overline k-\overline l)} + \right.\\
	&\left.+(\overline k+\overline l)\overline{\rho}_{\overline l}\frac{\partial\overline{q}^{(\overline k+\overline l)}}{\partial z}
		+(\overline k-\overline l)\overline{\rho}_{\overline l}\frac{\partial\overline{q}^{(\overline k-\overline l)}}{\partial z}\right)
		+ \overline{f}_x^{\overline k} = \\
	&=\frac{1}{2} \cdot \left(
		-conv(\overline\rho, \overline{p})_{\overline k}
			-corr(\overline\rho, \overline{p})_{\overline k}
			+ conv(\overline\rho, \frac{\partial\overline{q}}{\partial z})_{\overline k} 
			+ corr(\overline\rho, \frac{\partial\overline{q}}{\partial z})_{\overline k} \right)
		+ \overline{f}_x^{\overline k} = \\
	&=\frac{1}{2} \cdot \left(
		-sum(\overline\rho, \overline{p})_{\overline k}
			+ sum(\overline\rho, \frac{\partial\overline{q}}{\partial z})_{\overline k} \right)
		+ \overline{f}_x^{\overline k}; \\
&\frac{\partial\overline{w}_{\overline k}}{\partial t}
	=\frac{1}{2} \cdot \left(
		sum(\overline\rho, \overline{q})_{\overline k}
			+ sum(\overline\rho, \frac{\partial\overline{s}}{\partial z})_{\overline k} \right)
		+ \overline{f}_z^{\overline k}; \longalignedtag\\
&\frac{\partial\overline{p}_{\overline k}}{\partial t}
	=\frac{1}{2} \cdot \left(
		sum((\overline{\lambda} + 2\overline{\mu}), \overline{u})_{\overline k}
			+ sum(\overline\lambda, \frac{\partial\overline{w}}{\partial z})_{\overline k} \right) ; \\
&\frac{\partial\overline{q}_{\overline k}}{\partial t}
	=\frac{1}{2} \cdot \left(
		sum(\overline\mu, \frac{\partial\overline{u}}{\partial z})_{\overline k}
			- sum(\overline\mu, \overline{w})_{\overline k} \right) ; \\
&\frac{\partial\overline{s}_{\overline k}}{\partial t}
	=\frac{1}{2} \cdot \left(
		sum((\overline{\lambda} + 2\overline{\mu}), \frac{\partial\overline{w}}{\partial z})_{\overline k}
			+ sum(\overline\lambda, \overline{u})_{\overline k} \right) ; \\
\end{longaligned}
Начальные и граничные условия примут вид
\begin{equation}
\begin{dcases}
\label{eq:8}
\overline u^{\overline k} = \overline w^{\overline k} = \overline p^{\overline k} = \overline q^{\overline k} = \overline s^{\overline k} = 0; \\
\overline p^{\overline k} = \overline s^{\overline k} = 0
\end{dcases}
\end{equation}

Формулы обращения для нахождения решения исходной системы (\ref{eq:1}-\ref{eq:3}) примут вид:
\begin{longaligned}
\label{eq:8_1}
{}
&u(x,z,t) = \frac{2}{b}\sum_{\overline k=1}^{\infty}\overline u_{\overline k}(k,z,t)sin(\overline kx) \\
&w(x,z,t) = \frac{1}{b}\overline w(0,z,t) + \frac{2}{b}\sum_{\overline k=1}^{\infty}\overline w_{\overline k}(k,z,t)cos(\overline kx) \\
&\sigma_{xx}(x,z,t) = \frac{1}{b}\overline \tau_{xx}(0,z,t)
	+ \frac{2}{b}\sum_{\overline k=1}^{\infty}\overline \sigma_{xx}^{\overline k}(k,z,t)cos(\overline kx) \\
&\sigma_{xz}(x,z,t) = \frac{2}{b}\sum_{\overline k=1}^{\infty}\overline \sigma_{xz}^{\overline k}(k,z,t)sin(\overline kx) \longalignedtag\\
&\sigma_{zz}(x,z,t) = \frac{1}{b}\overline \sigma_{zz}(0,z,t)
	+ \frac{2}{b}\sum_{\overline k=1}^{\infty}\overline \sigma_{zz}^{\overline k}(k,z,t)cos(\overline kx) \\
\end{longaligned}

\subsection{Метод решения: разностная схема для преобразованной задачи}
Новую, уже одномерную задачу (\ref{eq:7}-\ref{eq:8}), полученную из исходной постановки (\ref{eq:1}-\ref{eq:3}),
можно решать различными конечно-разностными методами. В данной работе в качестве разностной схемы для решения (\ref{eq:7}-\ref{eq:8})
был взят одномерный аналог, хорошо себя зарекомендовавшей, конечно-разностной схемы Верье на сдвинутых сетках [Virieux J. P-SV wave propagation in heterogeneous media: Velocity-stress finite-difference method // Geophysics, Volume 51, April 1986, Number 4, p. 889-901].
\addimghere{nodes}{0.30}{Расположение узлов сетки}{nodes}
Полагая, что $\tau$ - шаг по времени, а $h$ - по пространственной координате, получаем следующее конечно-разностное представление системы (\ref{eq:7}-\ref{eq:8})
\begin{longaligned}
\label{eq:9}
{}
&\frac{\overline u_{n+1,i}^{\overline k} - \overline u_{n,i}^{\overline k}}{\tau}
= \frac{1}{2} \left[sum(\frac{\overline q_{n,i+\frac{1}{2}}^{\overline k} - \overline q_{n,i-\frac{1}{2}}^{\overline k}}{h}, \overline\rho_i)_{\overline k} 
	- sum(\overline p_{n, i}, \overline\rho_i)_{\overline k} \right] \\
&\frac{\overline w_{n+1,i+\frac{1}{2}}^{\overline k} - \overline w_{n+1,i+\frac{1}{2}}^{\overline k}}{\tau} 
= \frac{1}{2} \left[sum(\overline q_{n, i+\frac{1}{2}}^{\overline k}, \overline\rho_{i+\frac{1}{2}})_{\overline k}
	- sum(\frac{\overline s_{n,i+1}^{\overline k} - \overline s_{n,i}^{\overline k}}{h}, \overline\rho_{i+\frac{1}{2}})_{\overline k} \right] \\
&\frac{\overline p_{n+1,i}^{\overline k} - \overline p_{n,i}^{\overline k}}{\tau}
= \frac{1}{2} \left[sum(\frac{\overline w_{n,i+\frac{1}{2}}^{\overline k} - \overline w_{n,i-\frac{1}{2}}^{\overline k}}{h}, \overline\lambda_i)_{\overline k} 
	- sum(\overline u_{n, i}, \overline\lambda_i + 2\overline\mu_i)_{\overline k} \right] + f_{x,n}^{\overline k}\longalignedtag\\
&\frac{\overline q_{n+1,i+\frac{1}{2}}^{\overline k} - \overline q_{n+1,i+\frac{1}{2}}^{\overline k}}{\tau} 
= \frac{1}{2} \left[sum(\frac{\overline u_{n,i+1}^{\overline k} - \overline u_{n,i}^{\overline k}}{h}, \overline\mu_{i+\frac{1}{2}})_{\overline k}
	- sum(\overline w_{n, i+\frac{1}{2}}^{\overline k}, \overline\mu_{i+\frac{1}{2}})_{\overline k} \right] \\
&\frac{\overline s_{n+1,i}^{\overline k} - \overline s_{n,i}^{\overline k}}{\tau}
= \frac{1}{2} \left[sum(\frac{\overline w_{n,i+\frac{1}{2}}^{\overline k} - \overline w_{n,i-\frac{1}{2}}^{\overline k}}{h}, \overline\lambda_i+2\overline\mu_i)_{\overline k} 
	- sum(\overline u_{n, i}, \overline\lambda_i)_{\overline k} \right] + f_{z,n}^{\overline k}\\
\end{longaligned}

Где функция источника аппроксимируется следующим образом:
$$
f_{x,n}^{\overline k}=f_{z,n}^{\overline k}=\left\{
  \begin{array}{ccc}
    f(t)cos(\overline k x_0) & , z=z_0 \\
    0 & , z\neq z_0
  \end{array}
\right.
$$

А граничные условия:
$$
\frac{\overline w_{n+1, \frac{1}{2}} - \overline w_{n, \frac{1}{2}}}{\tau}
= \frac{1}{2} sum(\frac{2\overline s_{n,1}^{\overline k}}{h}, \overline\rho_1)
$$

Для того, чтобы обеспечить сходимость представленной выше явной разностной схемы, также необходимо выполнение условий Куранта:
\begin{equation}
\label{eq:10}
\tau \leq \frac{h}{\max\limits_z\left({V_p}\right)}
\end{equation}

\clearpage
